
% Default to the notebook output style

    


% Inherit from the specified cell style.




    
\documentclass[11pt]{article}

    
    
    \usepackage[T1]{fontenc}
    % Nicer default font (+ math font) than Computer Modern for most use cases
    \usepackage{mathpazo}

    % Basic figure setup, for now with no caption control since it's done
    % automatically by Pandoc (which extracts ![](path) syntax from Markdown).
    \usepackage{graphicx}
    % We will generate all images so they have a width \maxwidth. This means
    % that they will get their normal width if they fit onto the page, but
    % are scaled down if they would overflow the margins.
    \makeatletter
    \def\maxwidth{\ifdim\Gin@nat@width>\linewidth\linewidth
    \else\Gin@nat@width\fi}
    \makeatother
    \let\Oldincludegraphics\includegraphics
    % Set max figure width to be 80% of text width, for now hardcoded.
    \renewcommand{\includegraphics}[1]{\Oldincludegraphics[width=.8\maxwidth]{#1}}
    % Ensure that by default, figures have no caption (until we provide a
    % proper Figure object with a Caption API and a way to capture that
    % in the conversion process - todo).
    \usepackage{caption}
    \DeclareCaptionLabelFormat{nolabel}{}
    \captionsetup{labelformat=nolabel}

    \usepackage{adjustbox} % Used to constrain images to a maximum size 
    \usepackage{xcolor} % Allow colors to be defined
    \usepackage{enumerate} % Needed for markdown enumerations to work
    \usepackage{geometry} % Used to adjust the document margins
    \usepackage{amsmath} % Equations
    \usepackage{amssymb} % Equations
    \usepackage{textcomp} % defines textquotesingle
    % Hack from http://tex.stackexchange.com/a/47451/13684:
    \AtBeginDocument{%
        \def\PYZsq{\textquotesingle}% Upright quotes in Pygmentized code
    }
    \usepackage{upquote} % Upright quotes for verbatim code
    \usepackage{eurosym} % defines \euro
    \usepackage[mathletters]{ucs} % Extended unicode (utf-8) support
    \usepackage[utf8x]{inputenc} % Allow utf-8 characters in the tex document
    \usepackage{fancyvrb} % verbatim replacement that allows latex
    \usepackage{grffile} % extends the file name processing of package graphics 
                         % to support a larger range 
    % The hyperref package gives us a pdf with properly built
    % internal navigation ('pdf bookmarks' for the table of contents,
    % internal cross-reference links, web links for URLs, etc.)
    \usepackage{hyperref}
    \usepackage{longtable} % longtable support required by pandoc >1.10
    \usepackage{booktabs}  % table support for pandoc > 1.12.2
    \usepackage[inline]{enumitem} % IRkernel/repr support (it uses the enumerate* environment)
    \usepackage[normalem]{ulem} % ulem is needed to support strikethroughs (\sout)
                                % normalem makes italics be italics, not underlines
    

    
    
    % Colors for the hyperref package
    \definecolor{urlcolor}{rgb}{0,.145,.698}
    \definecolor{linkcolor}{rgb}{.71,0.21,0.01}
    \definecolor{citecolor}{rgb}{.12,.54,.11}

    % ANSI colors
    \definecolor{ansi-black}{HTML}{3E424D}
    \definecolor{ansi-black-intense}{HTML}{282C36}
    \definecolor{ansi-red}{HTML}{E75C58}
    \definecolor{ansi-red-intense}{HTML}{B22B31}
    \definecolor{ansi-green}{HTML}{00A250}
    \definecolor{ansi-green-intense}{HTML}{007427}
    \definecolor{ansi-yellow}{HTML}{DDB62B}
    \definecolor{ansi-yellow-intense}{HTML}{B27D12}
    \definecolor{ansi-blue}{HTML}{208FFB}
    \definecolor{ansi-blue-intense}{HTML}{0065CA}
    \definecolor{ansi-magenta}{HTML}{D160C4}
    \definecolor{ansi-magenta-intense}{HTML}{A03196}
    \definecolor{ansi-cyan}{HTML}{60C6C8}
    \definecolor{ansi-cyan-intense}{HTML}{258F8F}
    \definecolor{ansi-white}{HTML}{C5C1B4}
    \definecolor{ansi-white-intense}{HTML}{A1A6B2}

    % commands and environments needed by pandoc snippets
    % extracted from the output of `pandoc -s`
    \providecommand{\tightlist}{%
      \setlength{\itemsep}{0pt}\setlength{\parskip}{0pt}}
    \DefineVerbatimEnvironment{Highlighting}{Verbatim}{commandchars=\\\{\}}
    % Add ',fontsize=\small' for more characters per line
    \newenvironment{Shaded}{}{}
    \newcommand{\KeywordTok}[1]{\textcolor[rgb]{0.00,0.44,0.13}{\textbf{{#1}}}}
    \newcommand{\DataTypeTok}[1]{\textcolor[rgb]{0.56,0.13,0.00}{{#1}}}
    \newcommand{\DecValTok}[1]{\textcolor[rgb]{0.25,0.63,0.44}{{#1}}}
    \newcommand{\BaseNTok}[1]{\textcolor[rgb]{0.25,0.63,0.44}{{#1}}}
    \newcommand{\FloatTok}[1]{\textcolor[rgb]{0.25,0.63,0.44}{{#1}}}
    \newcommand{\CharTok}[1]{\textcolor[rgb]{0.25,0.44,0.63}{{#1}}}
    \newcommand{\StringTok}[1]{\textcolor[rgb]{0.25,0.44,0.63}{{#1}}}
    \newcommand{\CommentTok}[1]{\textcolor[rgb]{0.38,0.63,0.69}{\textit{{#1}}}}
    \newcommand{\OtherTok}[1]{\textcolor[rgb]{0.00,0.44,0.13}{{#1}}}
    \newcommand{\AlertTok}[1]{\textcolor[rgb]{1.00,0.00,0.00}{\textbf{{#1}}}}
    \newcommand{\FunctionTok}[1]{\textcolor[rgb]{0.02,0.16,0.49}{{#1}}}
    \newcommand{\RegionMarkerTok}[1]{{#1}}
    \newcommand{\ErrorTok}[1]{\textcolor[rgb]{1.00,0.00,0.00}{\textbf{{#1}}}}
    \newcommand{\NormalTok}[1]{{#1}}
    
    % Additional commands for more recent versions of Pandoc
    \newcommand{\ConstantTok}[1]{\textcolor[rgb]{0.53,0.00,0.00}{{#1}}}
    \newcommand{\SpecialCharTok}[1]{\textcolor[rgb]{0.25,0.44,0.63}{{#1}}}
    \newcommand{\VerbatimStringTok}[1]{\textcolor[rgb]{0.25,0.44,0.63}{{#1}}}
    \newcommand{\SpecialStringTok}[1]{\textcolor[rgb]{0.73,0.40,0.53}{{#1}}}
    \newcommand{\ImportTok}[1]{{#1}}
    \newcommand{\DocumentationTok}[1]{\textcolor[rgb]{0.73,0.13,0.13}{\textit{{#1}}}}
    \newcommand{\AnnotationTok}[1]{\textcolor[rgb]{0.38,0.63,0.69}{\textbf{\textit{{#1}}}}}
    \newcommand{\CommentVarTok}[1]{\textcolor[rgb]{0.38,0.63,0.69}{\textbf{\textit{{#1}}}}}
    \newcommand{\VariableTok}[1]{\textcolor[rgb]{0.10,0.09,0.49}{{#1}}}
    \newcommand{\ControlFlowTok}[1]{\textcolor[rgb]{0.00,0.44,0.13}{\textbf{{#1}}}}
    \newcommand{\OperatorTok}[1]{\textcolor[rgb]{0.40,0.40,0.40}{{#1}}}
    \newcommand{\BuiltInTok}[1]{{#1}}
    \newcommand{\ExtensionTok}[1]{{#1}}
    \newcommand{\PreprocessorTok}[1]{\textcolor[rgb]{0.74,0.48,0.00}{{#1}}}
    \newcommand{\AttributeTok}[1]{\textcolor[rgb]{0.49,0.56,0.16}{{#1}}}
    \newcommand{\InformationTok}[1]{\textcolor[rgb]{0.38,0.63,0.69}{\textbf{\textit{{#1}}}}}
    \newcommand{\WarningTok}[1]{\textcolor[rgb]{0.38,0.63,0.69}{\textbf{\textit{{#1}}}}}
    
    
    % Define a nice break command that doesn't care if a line doesn't already
    % exist.
    \def\br{\hspace*{\fill} \\* }
    % Math Jax compatability definitions
	\def\TeX{\mbox{T\kern-.14em\lower.5ex\hbox{E}\kern-.115em X}}
	\def\LaTeX{\mbox{L\kern-.325em\raise.21em\hbox{$\scriptstyle{A}$}\kern-.17em}\TeX}

    \def\gt{>}
    \def\lt{<}
    % Document parameters
    \title{HW5-Interference}
    
    
    

    % Pygments definitions
    
\makeatletter
\def\PY@reset{\let\PY@it=\relax \let\PY@bf=\relax%
    \let\PY@ul=\relax \let\PY@tc=\relax%
    \let\PY@bc=\relax \let\PY@ff=\relax}
\def\PY@tok#1{\csname PY@tok@#1\endcsname}
\def\PY@toks#1+{\ifx\relax#1\empty\else%
    \PY@tok{#1}\expandafter\PY@toks\fi}
\def\PY@do#1{\PY@bc{\PY@tc{\PY@ul{%
    \PY@it{\PY@bf{\PY@ff{#1}}}}}}}
\def\PY#1#2{\PY@reset\PY@toks#1+\relax+\PY@do{#2}}

\expandafter\def\csname PY@tok@gd\endcsname{\def\PY@tc##1{\textcolor[rgb]{0.63,0.00,0.00}{##1}}}
\expandafter\def\csname PY@tok@gu\endcsname{\let\PY@bf=\textbf\def\PY@tc##1{\textcolor[rgb]{0.50,0.00,0.50}{##1}}}
\expandafter\def\csname PY@tok@gt\endcsname{\def\PY@tc##1{\textcolor[rgb]{0.00,0.27,0.87}{##1}}}
\expandafter\def\csname PY@tok@gs\endcsname{\let\PY@bf=\textbf}
\expandafter\def\csname PY@tok@gr\endcsname{\def\PY@tc##1{\textcolor[rgb]{1.00,0.00,0.00}{##1}}}
\expandafter\def\csname PY@tok@cm\endcsname{\let\PY@it=\textit\def\PY@tc##1{\textcolor[rgb]{0.25,0.50,0.50}{##1}}}
\expandafter\def\csname PY@tok@vg\endcsname{\def\PY@tc##1{\textcolor[rgb]{0.10,0.09,0.49}{##1}}}
\expandafter\def\csname PY@tok@vi\endcsname{\def\PY@tc##1{\textcolor[rgb]{0.10,0.09,0.49}{##1}}}
\expandafter\def\csname PY@tok@vm\endcsname{\def\PY@tc##1{\textcolor[rgb]{0.10,0.09,0.49}{##1}}}
\expandafter\def\csname PY@tok@mh\endcsname{\def\PY@tc##1{\textcolor[rgb]{0.40,0.40,0.40}{##1}}}
\expandafter\def\csname PY@tok@cs\endcsname{\let\PY@it=\textit\def\PY@tc##1{\textcolor[rgb]{0.25,0.50,0.50}{##1}}}
\expandafter\def\csname PY@tok@ge\endcsname{\let\PY@it=\textit}
\expandafter\def\csname PY@tok@vc\endcsname{\def\PY@tc##1{\textcolor[rgb]{0.10,0.09,0.49}{##1}}}
\expandafter\def\csname PY@tok@il\endcsname{\def\PY@tc##1{\textcolor[rgb]{0.40,0.40,0.40}{##1}}}
\expandafter\def\csname PY@tok@go\endcsname{\def\PY@tc##1{\textcolor[rgb]{0.53,0.53,0.53}{##1}}}
\expandafter\def\csname PY@tok@cp\endcsname{\def\PY@tc##1{\textcolor[rgb]{0.74,0.48,0.00}{##1}}}
\expandafter\def\csname PY@tok@gi\endcsname{\def\PY@tc##1{\textcolor[rgb]{0.00,0.63,0.00}{##1}}}
\expandafter\def\csname PY@tok@gh\endcsname{\let\PY@bf=\textbf\def\PY@tc##1{\textcolor[rgb]{0.00,0.00,0.50}{##1}}}
\expandafter\def\csname PY@tok@ni\endcsname{\let\PY@bf=\textbf\def\PY@tc##1{\textcolor[rgb]{0.60,0.60,0.60}{##1}}}
\expandafter\def\csname PY@tok@nl\endcsname{\def\PY@tc##1{\textcolor[rgb]{0.63,0.63,0.00}{##1}}}
\expandafter\def\csname PY@tok@nn\endcsname{\let\PY@bf=\textbf\def\PY@tc##1{\textcolor[rgb]{0.00,0.00,1.00}{##1}}}
\expandafter\def\csname PY@tok@no\endcsname{\def\PY@tc##1{\textcolor[rgb]{0.53,0.00,0.00}{##1}}}
\expandafter\def\csname PY@tok@na\endcsname{\def\PY@tc##1{\textcolor[rgb]{0.49,0.56,0.16}{##1}}}
\expandafter\def\csname PY@tok@nb\endcsname{\def\PY@tc##1{\textcolor[rgb]{0.00,0.50,0.00}{##1}}}
\expandafter\def\csname PY@tok@nc\endcsname{\let\PY@bf=\textbf\def\PY@tc##1{\textcolor[rgb]{0.00,0.00,1.00}{##1}}}
\expandafter\def\csname PY@tok@nd\endcsname{\def\PY@tc##1{\textcolor[rgb]{0.67,0.13,1.00}{##1}}}
\expandafter\def\csname PY@tok@ne\endcsname{\let\PY@bf=\textbf\def\PY@tc##1{\textcolor[rgb]{0.82,0.25,0.23}{##1}}}
\expandafter\def\csname PY@tok@nf\endcsname{\def\PY@tc##1{\textcolor[rgb]{0.00,0.00,1.00}{##1}}}
\expandafter\def\csname PY@tok@si\endcsname{\let\PY@bf=\textbf\def\PY@tc##1{\textcolor[rgb]{0.73,0.40,0.53}{##1}}}
\expandafter\def\csname PY@tok@s2\endcsname{\def\PY@tc##1{\textcolor[rgb]{0.73,0.13,0.13}{##1}}}
\expandafter\def\csname PY@tok@nt\endcsname{\let\PY@bf=\textbf\def\PY@tc##1{\textcolor[rgb]{0.00,0.50,0.00}{##1}}}
\expandafter\def\csname PY@tok@nv\endcsname{\def\PY@tc##1{\textcolor[rgb]{0.10,0.09,0.49}{##1}}}
\expandafter\def\csname PY@tok@s1\endcsname{\def\PY@tc##1{\textcolor[rgb]{0.73,0.13,0.13}{##1}}}
\expandafter\def\csname PY@tok@dl\endcsname{\def\PY@tc##1{\textcolor[rgb]{0.73,0.13,0.13}{##1}}}
\expandafter\def\csname PY@tok@ch\endcsname{\let\PY@it=\textit\def\PY@tc##1{\textcolor[rgb]{0.25,0.50,0.50}{##1}}}
\expandafter\def\csname PY@tok@m\endcsname{\def\PY@tc##1{\textcolor[rgb]{0.40,0.40,0.40}{##1}}}
\expandafter\def\csname PY@tok@gp\endcsname{\let\PY@bf=\textbf\def\PY@tc##1{\textcolor[rgb]{0.00,0.00,0.50}{##1}}}
\expandafter\def\csname PY@tok@sh\endcsname{\def\PY@tc##1{\textcolor[rgb]{0.73,0.13,0.13}{##1}}}
\expandafter\def\csname PY@tok@ow\endcsname{\let\PY@bf=\textbf\def\PY@tc##1{\textcolor[rgb]{0.67,0.13,1.00}{##1}}}
\expandafter\def\csname PY@tok@sx\endcsname{\def\PY@tc##1{\textcolor[rgb]{0.00,0.50,0.00}{##1}}}
\expandafter\def\csname PY@tok@bp\endcsname{\def\PY@tc##1{\textcolor[rgb]{0.00,0.50,0.00}{##1}}}
\expandafter\def\csname PY@tok@c1\endcsname{\let\PY@it=\textit\def\PY@tc##1{\textcolor[rgb]{0.25,0.50,0.50}{##1}}}
\expandafter\def\csname PY@tok@fm\endcsname{\def\PY@tc##1{\textcolor[rgb]{0.00,0.00,1.00}{##1}}}
\expandafter\def\csname PY@tok@o\endcsname{\def\PY@tc##1{\textcolor[rgb]{0.40,0.40,0.40}{##1}}}
\expandafter\def\csname PY@tok@kc\endcsname{\let\PY@bf=\textbf\def\PY@tc##1{\textcolor[rgb]{0.00,0.50,0.00}{##1}}}
\expandafter\def\csname PY@tok@c\endcsname{\let\PY@it=\textit\def\PY@tc##1{\textcolor[rgb]{0.25,0.50,0.50}{##1}}}
\expandafter\def\csname PY@tok@mf\endcsname{\def\PY@tc##1{\textcolor[rgb]{0.40,0.40,0.40}{##1}}}
\expandafter\def\csname PY@tok@err\endcsname{\def\PY@bc##1{\setlength{\fboxsep}{0pt}\fcolorbox[rgb]{1.00,0.00,0.00}{1,1,1}{\strut ##1}}}
\expandafter\def\csname PY@tok@mb\endcsname{\def\PY@tc##1{\textcolor[rgb]{0.40,0.40,0.40}{##1}}}
\expandafter\def\csname PY@tok@ss\endcsname{\def\PY@tc##1{\textcolor[rgb]{0.10,0.09,0.49}{##1}}}
\expandafter\def\csname PY@tok@sr\endcsname{\def\PY@tc##1{\textcolor[rgb]{0.73,0.40,0.53}{##1}}}
\expandafter\def\csname PY@tok@mo\endcsname{\def\PY@tc##1{\textcolor[rgb]{0.40,0.40,0.40}{##1}}}
\expandafter\def\csname PY@tok@kd\endcsname{\let\PY@bf=\textbf\def\PY@tc##1{\textcolor[rgb]{0.00,0.50,0.00}{##1}}}
\expandafter\def\csname PY@tok@mi\endcsname{\def\PY@tc##1{\textcolor[rgb]{0.40,0.40,0.40}{##1}}}
\expandafter\def\csname PY@tok@kn\endcsname{\let\PY@bf=\textbf\def\PY@tc##1{\textcolor[rgb]{0.00,0.50,0.00}{##1}}}
\expandafter\def\csname PY@tok@cpf\endcsname{\let\PY@it=\textit\def\PY@tc##1{\textcolor[rgb]{0.25,0.50,0.50}{##1}}}
\expandafter\def\csname PY@tok@kr\endcsname{\let\PY@bf=\textbf\def\PY@tc##1{\textcolor[rgb]{0.00,0.50,0.00}{##1}}}
\expandafter\def\csname PY@tok@s\endcsname{\def\PY@tc##1{\textcolor[rgb]{0.73,0.13,0.13}{##1}}}
\expandafter\def\csname PY@tok@kp\endcsname{\def\PY@tc##1{\textcolor[rgb]{0.00,0.50,0.00}{##1}}}
\expandafter\def\csname PY@tok@w\endcsname{\def\PY@tc##1{\textcolor[rgb]{0.73,0.73,0.73}{##1}}}
\expandafter\def\csname PY@tok@kt\endcsname{\def\PY@tc##1{\textcolor[rgb]{0.69,0.00,0.25}{##1}}}
\expandafter\def\csname PY@tok@sc\endcsname{\def\PY@tc##1{\textcolor[rgb]{0.73,0.13,0.13}{##1}}}
\expandafter\def\csname PY@tok@sb\endcsname{\def\PY@tc##1{\textcolor[rgb]{0.73,0.13,0.13}{##1}}}
\expandafter\def\csname PY@tok@sa\endcsname{\def\PY@tc##1{\textcolor[rgb]{0.73,0.13,0.13}{##1}}}
\expandafter\def\csname PY@tok@k\endcsname{\let\PY@bf=\textbf\def\PY@tc##1{\textcolor[rgb]{0.00,0.50,0.00}{##1}}}
\expandafter\def\csname PY@tok@se\endcsname{\let\PY@bf=\textbf\def\PY@tc##1{\textcolor[rgb]{0.73,0.40,0.13}{##1}}}
\expandafter\def\csname PY@tok@sd\endcsname{\let\PY@it=\textit\def\PY@tc##1{\textcolor[rgb]{0.73,0.13,0.13}{##1}}}

\def\PYZbs{\char`\\}
\def\PYZus{\char`\_}
\def\PYZob{\char`\{}
\def\PYZcb{\char`\}}
\def\PYZca{\char`\^}
\def\PYZam{\char`\&}
\def\PYZlt{\char`\<}
\def\PYZgt{\char`\>}
\def\PYZsh{\char`\#}
\def\PYZpc{\char`\%}
\def\PYZdl{\char`\$}
\def\PYZhy{\char`\-}
\def\PYZsq{\char`\'}
\def\PYZdq{\char`\"}
\def\PYZti{\char`\~}
% for compatibility with earlier versions
\def\PYZat{@}
\def\PYZlb{[}
\def\PYZrb{]}
\makeatother


    % Exact colors from NB
    \definecolor{incolor}{rgb}{0.0, 0.0, 0.5}
    \definecolor{outcolor}{rgb}{0.545, 0.0, 0.0}



    
    % Prevent overflowing lines due to hard-to-break entities
    \sloppy 
    % Setup hyperref package
    \hypersetup{
      breaklinks=true,  % so long urls are correctly broken across lines
      colorlinks=true,
      urlcolor=urlcolor,
      linkcolor=linkcolor,
      citecolor=citecolor,
      }
    % Slightly bigger margins than the latex defaults
    
    \geometry{verbose,tmargin=1in,bmargin=1in,lmargin=1in,rmargin=1in}
    
    

    \begin{document}
    
    
    \maketitle
    
    

    
    \section{Homework 5 - Interference of
Light}\label{homework-5---interference-of-light}

I'm going to use a Jupyter notebook to solve the homework so you can see
the utility of the Python notebook.

    \begin{Verbatim}[commandchars=\\\{\}]
{\color{incolor}In [{\color{incolor}1}]:} \PY{c+c1}{\PYZsh{} setup by importing some good modules}
        \PY{k+kn}{import} \PY{n+nn}{sympy} \PY{k}{as} \PY{n+nn}{sp}
        \PY{c+c1}{\PYZsh{} All calls to sympy require sp. at the beginning}
        \PY{c+c1}{\PYZsh{} One could use \PYZdq{}from sympy import *\PYZdq{}, but then it\PYZsq{}s difficult to know what is sympy}
        \PY{c+c1}{\PYZsh{} and what is another module when importing multiple modules.}
        \PY{c+c1}{\PYZsh{} print things all pretty}
        \PY{k+kn}{from} \PY{n+nn}{sympy}\PY{n+nn}{.}\PY{n+nn}{abc} \PY{k}{import} \PY{o}{*}
        \PY{n}{sp}\PY{o}{.}\PY{n}{init\PYZus{}printing}\PY{p}{(}\PY{n}{use\PYZus{}latex}\PY{o}{=}\PY{l+s+s1}{\PYZsq{}}\PY{l+s+s1}{mathjax}\PY{l+s+s1}{\PYZsq{}}\PY{p}{)}
\end{Verbatim}


    \paragraph{1. Define these terms in your own words and write a sentence
for each term explaining how a light source with this property assists
in observing interference and diffraction. Are they necessary for
observing interference and
diffraction?}\label{define-these-terms-in-your-own-words-and-write-a-sentence-for-each-term-explaining-how-a-light-source-with-this-property-assists-in-observing-interference-and-diffraction.-are-they-necessary-for-observing-interference-and-diffraction}

\begin{enumerate}
\def\labelenumi{\alph{enumi}.}
\tightlist
\item
  Monochromatic - consisting of one color. This relates to a light
  source emitting a single wavelength of light. This would be useful in
  observing interference and diffraction because it simplifies the
  resulting pattern to having a single set of maxima and minima related
  to the one wavelength. Multiple wavelengths can overlap or otherwise
  make it more difficult to see the patterns.
\item
  Coherent - all waves having the same phase. When waves have the same
  phase, they intefere in the same predictable way with one another when
  passing through slit(s). Non-coherent light would have more
  complicated patterns due to phase differences of light entering the
  slit(s).
\item
  Uniaxial - light traveling along a single direction. Lasers are a good
  example. They have a beam that travels uniaxially. Most light sources
  are more isotropic or multiaxial. It is more difficult to direct light
  through a slit or set of slits when it is going in all directions. It
  is also difficult to eliminate the "noise" of light that does not pass
  through slit(s) when it is not uniaxial.
\end{enumerate}

None of these properties are required for observing interference and
diffraction.

\paragraph{2. Your humble professor wants to perform a lecture
demonstration of the classic Young's double-slit experiment. As a light
source, he uses the 632.8-nm light from a He--Ne laser. The interference
pattern will be projected on a wall that is 3.0 m from the slits. For
easy viewing by all students in the class, he wants the distance between
the m=0 and m=1 maxima to be 25 cm. What slit separation is required in
order to produce the desired interference
pattern?}\label{your-humble-professor-wants-to-perform-a-lecture-demonstration-of-the-classic-youngs-double-slit-experiment.-as-a-light-source-he-uses-the-632.8-nm-light-from-a-hene-laser.-the-interference-pattern-will-be-projected-on-a-wall-that-is-3.0-m-from-the-slits.-for-easy-viewing-by-all-students-in-the-class-he-wants-the-distance-between-the-m0-and-m1-maxima-to-be-25-cm.-what-slit-separation-is-required-in-order-to-produce-the-desired-interference-pattern}

We are looking for \(\theta_1\) in the image below. The image is
actually incorrect. The angle should be made where the spot is on the
screen, not at the maximum amplitude of the pattern away from the
screen. The small angle approximation will work since
\(\frac{0.25}{3} = 0.083\) radians. This is about 6 degrees.
\[\begin{equation}
d \sin\theta = m\lambda\\
\\
d \frac{y}{L} = m\lambda\\
\\
d = \frac{1 \left(632.8\times10^{-9}\right)\left(3\right)}{0.25}\\
\\
d = 7.6\times10^{-6} \rm{m}
\end{equation}\]

    \begin{Verbatim}[commandchars=\\\{\}]
{\color{incolor}In [{\color{incolor}70}]:} \PY{c+c1}{\PYZsh{}Let\PYZsq{}s graph the pattern as a function of theta}
         \PY{c+c1}{\PYZsh{}Create your constants}
         \PY{n}{wl} \PY{o}{=} \PY{l+m+mf}{632.8e\PYZhy{}9} \PY{c+c1}{\PYZsh{}variable for wavelength}
         \PY{n}{D} \PY{o}{=} \PY{l+m+mi}{2}\PY{o}{*}\PY{n}{wl} \PY{c+c1}{\PYZsh{}variable for slit width}
         \PY{n}{dd} \PY{o}{=} \PY{l+m+mf}{2.53e\PYZhy{}6}
         \PY{n}{L} \PY{o}{=} \PY{l+m+mi}{3} \PY{c+c1}{\PYZsh{}variable for screen distance from slit}
         \PY{c+c1}{\PYZsh{}Define x and I as real\PYZhy{}valued symbols}
         \PY{n}{x}\PY{p}{,} \PY{n}{I}\PY{o}{=} \PY{n}{symbols}\PY{p}{(}\PY{l+s+s2}{\PYZdq{}}\PY{l+s+s2}{x, I}\PY{l+s+s2}{\PYZdq{}}\PY{p}{,} \PY{n}{real} \PY{o}{=} \PY{k+kc}{True}\PY{p}{)}
         \PY{c+c1}{\PYZsh{} Define I for real \PYZhy{}\PYZhy{} making x be in degrees}
         \PY{n}{I} \PY{o}{=} \PY{p}{(}\PY{n}{sp}\PY{o}{.}\PY{n}{sin}\PY{p}{(}\PY{n}{sp}\PY{o}{.}\PY{n}{pi}\PY{o}{*}\PY{n}{D}\PY{o}{*}\PY{n}{sp}\PY{o}{.}\PY{n}{sin}\PY{p}{(}\PY{n}{x}\PY{o}{*}\PY{n}{sp}\PY{o}{.}\PY{n}{pi}\PY{o}{/}\PY{l+m+mi}{180}\PY{p}{)}\PY{o}{/}\PY{n}{wl}\PY{p}{)}\PY{o}{/}\PY{p}{(}\PY{n}{sp}\PY{o}{.}\PY{n}{pi}\PY{o}{*}\PY{n}{D}\PY{o}{*}\PY{n}{sp}\PY{o}{.}\PY{n}{sin}\PY{p}{(}\PY{n}{sp}\PY{o}{.}\PY{n}{pi}\PY{o}{*}\PY{n}{x}\PY{o}{/}\PY{l+m+mi}{180}\PY{p}{)}\PY{o}{/}\PY{n}{wl}\PY{p}{)}\PY{p}{)}\PY{o}{*}\PY{o}{*}\PY{l+m+mi}{2} \PY{o}{*} \PY{n}{sp}\PY{o}{.}\PY{n}{cos}\PY{p}{(}\PY{p}{(}\PY{n}{sp}\PY{o}{.}\PY{n}{pi}\PY{o}{*}\PY{n}{dd}\PY{o}{*}\PY{n}{sp}\PY{o}{.}\PY{n}{sin}\PY{p}{(}\PY{n}{sp}\PY{o}{.}\PY{n}{pi}\PY{o}{*}\PY{n}{x}\PY{o}{/}\PY{l+m+mi}{180}\PY{p}{)}\PY{o}{/}\PY{n}{wl}\PY{p}{)}\PY{p}{)}\PY{o}{*}\PY{o}{*}\PY{l+m+mi}{2}
         \PY{n}{I} \PY{c+c1}{\PYZsh{} This causes it to output formatted nicely. The print command formats it as computer code}
\end{Verbatim}

\texttt{\color{outcolor}Out[{\color{outcolor}70}]:}
    
    $$\frac{0.25 \cos^{2}{\left (3.99810366624526 \pi \sin{\left (\frac{\pi x}{180} \right )} \right )}}{\pi^{2} \sin^{2}{\left (\frac{\pi x}{180} \right )}} \sin^{2}{\left (2.0 \pi \sin{\left (\frac{\pi x}{180} \right )} \right )}$$

    

    \begin{Verbatim}[commandchars=\\\{\}]
{\color{incolor}In [{\color{incolor}71}]:} \PY{n}{sp}\PY{o}{.}\PY{n}{plot}\PY{p}{(}\PY{n}{I}\PY{p}{,} \PY{p}{(}\PY{n}{x}\PY{p}{,}\PY{o}{\PYZhy{}}\PY{l+m+mi}{30}\PY{p}{,} \PY{l+m+mi}{30}\PY{p}{)}\PY{p}{,} \PY{n}{xlabel}\PY{o}{=}\PY{l+s+s1}{\PYZsq{}}\PY{l+s+s1}{\PYZdl{}}\PY{l+s+se}{\PYZbs{}\PYZbs{}}\PY{l+s+s1}{theta\PYZdl{}}\PY{l+s+s1}{\PYZsq{}}\PY{p}{,} \PY{n}{ylabel}\PY{o}{=}\PY{l+s+s1}{\PYZsq{}}\PY{l+s+s1}{Intensity}\PY{l+s+s1}{\PYZsq{}}\PY{p}{)}
\end{Verbatim}


    \begin{center}
    \adjustimage{max size={0.9\linewidth}{0.9\paperheight}}{output_4_0.png}
    \end{center}
    { \hspace*{\fill} \\}
    
\begin{Verbatim}[commandchars=\\\{\}]
{\color{outcolor}Out[{\color{outcolor}71}]:} <sympy.plotting.plot.Plot at 0x7fdf2c902240>
\end{Verbatim}
            
    \paragraph{3. Light is described as monochromatic if it is composed of a
single wavelength. A laser emits monochromatic light. Imagine that such
a monochromatic source of light that passes through two narrow slits and
forms an interference pattern on a screen. (a) If the wavelength of
light used increases, will the distance between the maxima (1) increase,
(2) remain the same, or (3) decrease? Explain. (b) If the slit
separation is 0.25 mm, the screen is 1.5 m away from the slits, and
light of wavelength 550 nm is used, what is the distance from the center
of the central maximum to the center of the third-order maximum? (c)
What if the wavelength of light is 680
nm?}\label{light-is-described-as-monochromatic-if-it-is-composed-of-a-single-wavelength.-a-laser-emits-monochromatic-light.-imagine-that-such-a-monochromatic-source-of-light-that-passes-through-two-narrow-slits-and-forms-an-interference-pattern-on-a-screen.-a-if-the-wavelength-of-light-used-increases-will-the-distance-between-the-maxima-1-increase-2-remain-the-same-or-3-decrease-explain.-b-if-the-slit-separation-is-0.25-mm-the-screen-is-1.5-m-away-from-the-slits-and-light-of-wavelength-550-nm-is-used-what-is-the-distance-from-the-center-of-the-central-maximum-to-the-center-of-the-third-order-maximum-c-what-if-the-wavelength-of-light-is-680-nm}

\begin{enumerate}
\def\labelenumi{(\alph{enumi})}
\item
  Since the constructive interference condition is
  \(d \sin\theta = m\lambda\), the angle will increase with wavelength,
  assuming \(d\) is constant. This is because \(\sin\) is increasing
  over the range of 0 to 90 degrees where these maxima occur. This means
  the spacing increases for the maxima.
\item
  \[d\begin{equation}
  \sin\theta = m\lambda\\
  0.25\times10^{-3} \sin\theta = 3\left(550\times10^{-9}\right)\\
  \sin\theta = 6.6\times10^{-3}\\
  \sin\theta \approx \frac{y}{L}\\
  \frac{y}{L} \approx 6.6\times10^{-3}\\
  y\approx 6.6\times10^{-3}\left(1.5\right)\\
  y\approx 0.00099 = 9.9 \rm{mm}
  \end{equation}\]
\end{enumerate}

(c)\[d\begin{equation}
\sin\theta = m\lambda\\
0.25\times10^{-3} \sin\theta = 3\left(680\times10^{-9}\right)\\
\sin\theta = 8.16\times10^{-3}\\
\sin\theta \approx \frac{y}{L}\\
\frac{y}{L} \approx 8.16\times10^{-3}\\
y\approx 8.16\times10^{-3}\left(1.5\right)\\
y\approx 0.01224 = 1.224 \rm{cm}
\end{equation}\]

    \begin{Verbatim}[commandchars=\\\{\}]
{\color{incolor}In [{\color{incolor}3}]:} \PY{c+c1}{\PYZsh{}Let\PYZsq{}s graph the pattern as a function of theta}
        \PY{c+c1}{\PYZsh{}Create your constants}
        \PY{n}{wl1} \PY{o}{=} \PY{l+m+mf}{550e\PYZhy{}9} \PY{c+c1}{\PYZsh{}variable for wavelength}
        \PY{n}{wl2} \PY{o}{=} \PY{l+m+mf}{680e\PYZhy{}9}
        \PY{n}{D} \PY{o}{=} \PY{l+m+mi}{5}\PY{o}{*}\PY{n}{wl1} \PY{c+c1}{\PYZsh{}variable for slit width}
        \PY{n}{dd} \PY{o}{=} \PY{l+m+mf}{0.25E\PYZhy{}3}
        \PY{n}{L} \PY{o}{=} \PY{l+m+mf}{1.5} \PY{c+c1}{\PYZsh{}variable for screen distance from slit}
        \PY{c+c1}{\PYZsh{}Define x and I as real\PYZhy{}valued symbols}
        \PY{n}{x}\PY{p}{,} \PY{n}{I1}\PY{p}{,} \PY{n}{I2}\PY{o}{=} \PY{n}{symbols}\PY{p}{(}\PY{l+s+s2}{\PYZdq{}}\PY{l+s+s2}{x, I1, I2}\PY{l+s+s2}{\PYZdq{}}\PY{p}{,} \PY{n}{real} \PY{o}{=} \PY{k+kc}{True}\PY{p}{)}
        \PY{c+c1}{\PYZsh{} Define I for real \PYZhy{}\PYZhy{} making x be in degrees}
        \PY{n}{I1} \PY{o}{=} \PY{p}{(}\PY{n}{sp}\PY{o}{.}\PY{n}{sin}\PY{p}{(}\PY{n}{sp}\PY{o}{.}\PY{n}{pi}\PY{o}{*}\PY{n}{D}\PY{o}{*}\PY{n}{x}\PY{o}{/}\PY{p}{(}\PY{n}{L}\PY{o}{*}\PY{n}{wl1}\PY{p}{)}\PY{p}{)}\PY{o}{/}\PY{p}{(}\PY{n}{sp}\PY{o}{.}\PY{n}{pi}\PY{o}{*}\PY{n}{D}\PY{o}{*}\PY{n}{x}\PY{o}{/}\PY{p}{(}\PY{n}{L}\PY{o}{*}\PY{n}{wl1}\PY{p}{)}\PY{p}{)}\PY{p}{)}\PY{o}{*}\PY{o}{*}\PY{l+m+mi}{2} \PY{o}{*} \PY{n}{sp}\PY{o}{.}\PY{n}{cos}\PY{p}{(}\PY{n}{sp}\PY{o}{.}\PY{n}{pi}\PY{o}{*}\PY{n}{dd}\PY{o}{*}\PY{n}{x}\PY{o}{/}\PY{p}{(}\PY{n}{L}\PY{o}{*}\PY{n}{wl1}\PY{p}{)}\PY{p}{)}\PY{o}{*}\PY{o}{*}\PY{l+m+mi}{2}
        \PY{n}{I2} \PY{o}{=} \PY{p}{(}\PY{n}{sp}\PY{o}{.}\PY{n}{sin}\PY{p}{(}\PY{n}{sp}\PY{o}{.}\PY{n}{pi}\PY{o}{*}\PY{n}{D}\PY{o}{*}\PY{n}{x}\PY{o}{/}\PY{p}{(}\PY{n}{L}\PY{o}{*}\PY{n}{wl2}\PY{p}{)}\PY{p}{)}\PY{o}{/}\PY{p}{(}\PY{n}{sp}\PY{o}{.}\PY{n}{pi}\PY{o}{*}\PY{n}{D}\PY{o}{*}\PY{n}{x}\PY{o}{/}\PY{p}{(}\PY{n}{L}\PY{o}{*}\PY{n}{wl2}\PY{p}{)}\PY{p}{)}\PY{p}{)}\PY{o}{*}\PY{o}{*}\PY{l+m+mi}{2} \PY{o}{*} \PY{n}{sp}\PY{o}{.}\PY{n}{cos}\PY{p}{(}\PY{n}{sp}\PY{o}{.}\PY{n}{pi}\PY{o}{*}\PY{n}{dd}\PY{o}{*}\PY{n}{x}\PY{o}{/}\PY{p}{(}\PY{n}{L}\PY{o}{*}\PY{n}{wl2}\PY{p}{)}\PY{p}{)}\PY{o}{*}\PY{o}{*}\PY{l+m+mi}{2}
        
        \PY{n}{p} \PY{o}{=} \PY{n}{sp}\PY{o}{.}\PY{n}{plot}\PY{p}{(}\PY{n}{I1}\PY{p}{,} \PY{n}{I2}\PY{p}{,} \PY{p}{(}\PY{n}{x}\PY{p}{,}\PY{o}{\PYZhy{}}\PY{l+m+mf}{0.015}\PY{p}{,}\PY{l+m+mf}{0.015}\PY{p}{)}\PY{p}{,} \PY{n}{show}\PY{o}{=}\PY{k+kc}{False}\PY{p}{,} \PY{n}{legend}\PY{o}{=}\PY{k+kc}{True}\PY{p}{,} \PY{n}{xlabel}\PY{o}{=}\PY{l+s+s1}{\PYZsq{}}\PY{l+s+s1}{x (m)}\PY{l+s+s1}{\PYZsq{}}\PY{p}{,} \PY{n}{ylabel}\PY{o}{=}\PY{l+s+s1}{\PYZsq{}}\PY{l+s+s1}{Intensity}\PY{l+s+s1}{\PYZsq{}}\PY{p}{,} \PY{n}{ylim}\PY{o}{=}\PY{p}{(}\PY{l+m+mi}{0}\PY{p}{,}\PY{l+m+mi}{1}\PY{p}{)}\PY{p}{)}
        \PY{n}{p}\PY{p}{[}\PY{l+m+mi}{0}\PY{p}{]}\PY{o}{.}\PY{n}{line\PYZus{}color}\PY{o}{=}\PY{l+s+s1}{\PYZsq{}}\PY{l+s+s1}{green}\PY{l+s+s1}{\PYZsq{}}
        \PY{n}{p}\PY{p}{[}\PY{l+m+mi}{0}\PY{p}{]}\PY{o}{.}\PY{n}{label}\PY{o}{=}\PY{l+s+s1}{\PYZsq{}}\PY{l+s+s1}{550 nm}\PY{l+s+s1}{\PYZsq{}}
        \PY{n}{p}\PY{p}{[}\PY{l+m+mi}{1}\PY{p}{]}\PY{o}{.}\PY{n}{line\PYZus{}color}\PY{o}{=}\PY{l+s+s1}{\PYZsq{}}\PY{l+s+s1}{red}\PY{l+s+s1}{\PYZsq{}}
        \PY{n}{p}\PY{p}{[}\PY{l+m+mi}{1}\PY{p}{]}\PY{o}{.}\PY{n}{label}\PY{o}{=}\PY{l+s+s1}{\PYZsq{}}\PY{l+s+s1}{680 nm}\PY{l+s+s1}{\PYZsq{}}
        \PY{n}{p}\PY{o}{.}\PY{n}{show}\PY{p}{(}\PY{p}{)}
\end{Verbatim}


    \begin{center}
    \adjustimage{max size={0.9\linewidth}{0.9\paperheight}}{output_6_0.png}
    \end{center}
    { \hspace*{\fill} \\}
    
    We can see the third order maximum for the green 550 nm is very close to
0.01 m, while the red 680 nm third order maximum is between 0.01 and
0.015 m.

    \paragraph{4. The double-slit experiment was performed originally by
Thomas Young in 1801. However, at the time lasers did not exist and
light that was used was not monochromatic. Imagine recreating Young's
classic double-slit experiment with the light from a blue LED (also not
available in 1801) with a wavelength 480 nm (a bluish color). It is then
found that it gives a second-order maximum at a certain location on the
screen. What wavelengths of visible light would have a minimum at the
same location? Assume visible light is 350 nm to 900
nm.}\label{the-double-slit-experiment-was-performed-originally-by-thomas-young-in-1801.-however-at-the-time-lasers-did-not-exist-and-light-that-was-used-was-not-monochromatic.-imagine-recreating-youngs-classic-double-slit-experiment-with-the-light-from-a-blue-led-also-not-available-in-1801-with-a-wavelength-480-nm-a-bluish-color.-it-is-then-found-that-it-gives-a-second-order-maximum-at-a-certain-location-on-the-screen.-what-wavelengths-of-visible-light-would-have-a-minimum-at-the-same-location-assume-visible-light-is-350-nm-to-900-nm.}

We're looking for the second order maximum location of 480 nm light.
\[d\sin\theta=m\lambda\] where \(m=2\) and \(\lambda=480\times10^{-9}\)
m. We get the relationship \[d\sin\theta=9.6\times10^{-7}\] For these
maxima, \(d\) is the same for all wavelengths. The question is what
wavelengths between 350 and 900 nm will give \(9.6\times10^{-7}\) when
multiplied by a half-integer \(m+1/2\)?

\[\left(m+\frac{1}{2}\right)\lambda = 9.6\times10^{-7}\]

If \(m=1\), we get
\(\lambda = 9.6\times10^{-7}/1.5 = 6.40\times10^{-7} = 640\) nm. This is
red.

If \(m=2\), we get
\(\lambda = 4.8\times10^{-7}/2.5 = 3.84\times10^{-7} = 384\) nm. This is
blue.

If \(m=3\), we get
\(\lambda = 3.2\times10^{-7}/3.5 = 2.74\times10^{-7} = 274\) nm. This is
outside the range.

We can also look to see which wavelengths have a maximum at the same
location as our 2nd order maximum of 480 nm.

\[m\lambda = 9.6\times10^{-7}\]

If \(m=1\), we get
\(\lambda = 9.6\times10^{-7}/1 = 9.60\times10^{-7} = 960\) nm. This is
outside the visible range.

If \(m=2\), we get
\(\lambda = 9.6\times10^{-7}/2 = 4.80\times10^{-7} = 480\) nm. This is
our LED.

If \(m=3\), we get
\(\lambda = 9.6\times10^{-7}/3 = 3.2\times10^{-7} = 320\) nm. This is
outside the visible range.

So, we see that using 480 nm light is a good idea because there are no
other visible wavelengths that might affect our measurement.

    \paragraph{5. If you double the width of a single slit, the amplitude of
the light passing through the slit is doubled. (a) Show, however, that
the intensity at the center of the screen increases by a factor of 4.
(b) Explain why this does not violate conservation of
energy.}\label{if-you-double-the-width-of-a-single-slit-the-amplitude-of-the-light-passing-through-the-slit-is-doubled.-a-show-however-that-the-intensity-at-the-center-of-the-screen-increases-by-a-factor-of-4.-b-explain-why-this-does-not-violate-conservation-of-energy.}

\begin{enumerate}
\def\labelenumi{(\alph{enumi})}
\tightlist
\item
  The amplitude of light is related to the electric field, and the
  intensity is related to the electric field squared.
\end{enumerate}

\[E(x, t) = E_{\circ} \sin\left(kx-\omega t\right)\]

\[I(x, t) = E_{\circ}^2 \sin^2\left(kx-\omega t\right)\]

This means a particular point on the diffraction pattern has an
intensity related to \(E_{\circ}^2\). If the amplitude is doubled
because twice as many light rays go through, the intensity will be the
square of double the amplitude.

\[E_{\circ} \to 2E_{\circ}\]
\[I \to \left(2E_{\circ}\right)^2 = 4E_{\circ}^2\]

\begin{enumerate}
\def\labelenumi{(\alph{enumi})}
\setcounter{enumi}{1}
\tightlist
\item
  Four times the intensity arrives at the screen when the slit has
  double the width. Recall that intensity is Watts per meter squared
  \(\left(W/m^2 = Joules/s/m^2\right)\). This means the energy is four
  times as much during any interval of time in any of the constructive
  maxima peaks. It seems that we have violated conservation of energy
  because we allowed double the energy to enter the slit, and four times
  the energy shows up on the screen. Even though four times as much
  intensity shows up, the width of the pattern changes with the slit
  width.
\end{enumerate}

\[I \approx I_{\circ}\left[\frac{sin\left(\frac{\pi D y}{L\lambda}\right)}{\frac{\pi D y}{L\lambda}}\right]^2\]

This shows that the intensity pattern is inversely proportional to
\(D\). As \(D\) increases by a factor of two, the intensity pattern
decreases in area by a factor of two. Let's take a look quantitatively.
I'll create two intensity patterns with the relative slit widths being a
factor of two different.

    \begin{Verbatim}[commandchars=\\\{\}]
{\color{incolor}In [{\color{incolor}4}]:} \PY{c+c1}{\PYZsh{}Define x and intensities as real\PYZhy{}valued symbols}
        \PY{n}{x}\PY{p}{,} \PY{n}{Idiffract1}\PY{p}{,} \PY{n}{Idiffract2}\PY{o}{=} \PY{n}{symbols}\PY{p}{(}\PY{l+s+s2}{\PYZdq{}}\PY{l+s+s2}{x, Idiffract1, Idiffract2}\PY{l+s+s2}{\PYZdq{}}\PY{p}{,} \PY{n}{real} \PY{o}{=} \PY{k+kc}{True}\PY{p}{)}
        \PY{c+c1}{\PYZsh{} Define small angle intensity due to diffraction}
        \PY{c+c1}{\PYZsh{} Some variables wl1, L, and D were defined above}
        \PY{c+c1}{\PYZsh{} If Eo = 1, then (2Eo)\PYZca{}2=4. I can always find a scale where this is true.}
        \PY{n}{Idiffract1} \PY{o}{=} \PY{p}{(}\PY{n}{sp}\PY{o}{.}\PY{n}{sin}\PY{p}{(}\PY{n}{sp}\PY{o}{.}\PY{n}{pi}\PY{o}{*}\PY{n}{D}\PY{o}{*}\PY{n}{sp}\PY{o}{.}\PY{n}{sin}\PY{p}{(}\PY{n}{x}\PY{p}{)}\PY{o}{/}\PY{n}{wl1}\PY{p}{)}\PY{o}{/}\PY{p}{(}\PY{n}{sp}\PY{o}{.}\PY{n}{pi}\PY{o}{*}\PY{n}{D}\PY{o}{*}\PY{n}{sp}\PY{o}{.}\PY{n}{sin}\PY{p}{(}\PY{n}{x}\PY{p}{)}\PY{o}{/}\PY{n}{wl1}\PY{p}{)}\PY{p}{)}\PY{o}{*}\PY{o}{*}\PY{l+m+mi}{2}
        \PY{n}{Idiffract2} \PY{o}{=} \PY{l+m+mi}{4}\PY{o}{*}\PY{p}{(}\PY{n}{sp}\PY{o}{.}\PY{n}{sin}\PY{p}{(}\PY{n}{sp}\PY{o}{.}\PY{n}{pi}\PY{o}{*}\PY{l+m+mi}{2}\PY{o}{*}\PY{n}{D}\PY{o}{*}\PY{n}{sp}\PY{o}{.}\PY{n}{sin}\PY{p}{(}\PY{n}{x}\PY{p}{)}\PY{o}{/}\PY{n}{wl1}\PY{p}{)}\PY{o}{/}\PY{p}{(}\PY{n}{sp}\PY{o}{.}\PY{n}{pi}\PY{o}{*}\PY{l+m+mi}{2}\PY{o}{*}\PY{n}{D}\PY{o}{*}\PY{n}{sp}\PY{o}{.}\PY{n}{sin}\PY{p}{(}\PY{n}{x}\PY{p}{)}\PY{o}{/}\PY{n}{wl1}\PY{p}{)}\PY{p}{)}\PY{o}{*}\PY{o}{*}\PY{l+m+mi}{2}
        
        \PY{n}{p2} \PY{o}{=} \PY{n}{sp}\PY{o}{.}\PY{n}{plot}\PY{p}{(}\PY{n}{Idiffract1}\PY{p}{,} \PY{n}{Idiffract2}\PY{p}{,} \PY{p}{(}\PY{n}{x}\PY{p}{,}\PY{o}{\PYZhy{}}\PY{l+m+mf}{0.3}\PY{p}{,}\PY{l+m+mf}{0.3}\PY{p}{)}\PY{p}{,} \PY{n}{show}\PY{o}{=}\PY{k+kc}{False}\PY{p}{,} \PY{n}{legend}\PY{o}{=}\PY{k+kc}{True}\PY{p}{,} \PY{n}{xlabel}\PY{o}{=}\PY{l+s+s1}{\PYZsq{}}\PY{l+s+s1}{\PYZdl{}}\PY{l+s+se}{\PYZbs{}\PYZbs{}}\PY{l+s+s1}{theta\PYZdl{} (rad)}\PY{l+s+s1}{\PYZsq{}}\PY{p}{,} \PY{n}{ylabel}\PY{o}{=}\PY{l+s+s1}{\PYZsq{}}\PY{l+s+s1}{Intensity}\PY{l+s+s1}{\PYZsq{}}\PY{p}{)}
        \PY{n}{p2}\PY{p}{[}\PY{l+m+mi}{0}\PY{p}{]}\PY{o}{.}\PY{n}{line\PYZus{}color}\PY{o}{=}\PY{l+s+s1}{\PYZsq{}}\PY{l+s+s1}{green}\PY{l+s+s1}{\PYZsq{}}
        \PY{n}{p2}\PY{p}{[}\PY{l+m+mi}{0}\PY{p}{]}\PY{o}{.}\PY{n}{label}\PY{o}{=}\PY{l+s+s1}{\PYZsq{}}\PY{l+s+s1}{D}\PY{l+s+s1}{\PYZsq{}}
        \PY{n}{p2}\PY{p}{[}\PY{l+m+mi}{1}\PY{p}{]}\PY{o}{.}\PY{n}{line\PYZus{}color}\PY{o}{=}\PY{l+s+s1}{\PYZsq{}}\PY{l+s+s1}{red}\PY{l+s+s1}{\PYZsq{}}
        \PY{n}{p2}\PY{p}{[}\PY{l+m+mi}{1}\PY{p}{]}\PY{o}{.}\PY{n}{label}\PY{o}{=}\PY{l+s+s1}{\PYZsq{}}\PY{l+s+s1}{2D}\PY{l+s+s1}{\PYZsq{}}
        \PY{n}{p2}\PY{o}{.}\PY{n}{show}\PY{p}{(}\PY{p}{)}
\end{Verbatim}


    \begin{center}
    \adjustimage{max size={0.9\linewidth}{0.9\paperheight}}{output_10_0.png}
    \end{center}
    { \hspace*{\fill} \\}
    
    I know that the condition for a diffraction minimum is
\[D \sin\theta = m\lambda\] For the example graphed above, I used a slit
of \(D=5\lambda\) and \(\lambda=550\) nm
\[\theta_D = \sin^{-1}\left[\frac{1\left(550\times10^{-9}\right)}{5\times550\times10^{-9}}\right] = \sin^{-1}\left(\frac{1}{5}\right) = 0.201~\rm{rad} \]

The wider slit is \(2D\)
\[\theta_{2D} = \sin^{-1}\left[\frac{1\left(550\times10^{-9}\right)}{10\times550\times10^{-9}}\right] = \sin^{-1}\left(\frac{1}{10}\right) = 0.100~\rm{rad} \]

So, it's pretty obvious that the width is half, but what about the
integral to get the total intensity over the central peak?

    \begin{Verbatim}[commandchars=\\\{\}]
{\color{incolor}In [{\color{incolor}74}]:} \PY{n}{sp}\PY{o}{.}\PY{n}{integrate}\PY{p}{(}\PY{n}{Idiffract1}\PY{p}{,}\PY{p}{(}\PY{n}{x}\PY{p}{,}\PY{o}{\PYZhy{}}\PY{l+m+mf}{0.201}\PY{p}{,}\PY{l+m+mf}{0.201}\PY{p}{)}\PY{p}{)}
\end{Verbatim}

\texttt{\color{outcolor}Out[{\color{outcolor}74}]:}
    
    $$\frac{0.04}{\pi^{2}} \int_{-0.201}^{0.201} \frac{1}{\sin^{2}{\left (x \right )}} \sin^{2}{\left (5.0 \pi \sin{\left (x \right )} \right )}\, dx$$

    

    The Sympy \texttt{integrate} function formats the integral, but it's
unable to compute it. There is another SymPy function called
\texttt{Integral().evalf()} that will numerically compute the integral.
Let's see what we get.

    \begin{Verbatim}[commandchars=\\\{\}]
{\color{incolor}In [{\color{incolor}66}]:} \PY{n}{sp}\PY{o}{.}\PY{n}{Integral}\PY{p}{(}\PY{n}{Idiffract1}\PY{p}{,} \PY{p}{(}\PY{n}{x}\PY{p}{,}\PY{o}{\PYZhy{}}\PY{l+m+mf}{0.201}\PY{p}{,}\PY{l+m+mf}{0.201}\PY{p}{)}\PY{p}{)}\PY{o}{.}\PY{n}{evalf}\PY{p}{(}\PY{p}{)}
\end{Verbatim}

\texttt{\color{outcolor}Out[{\color{outcolor}66}]:}
    
    $$0.18$$

    

    \begin{Verbatim}[commandchars=\\\{\}]
{\color{incolor}In [{\color{incolor}67}]:} \PY{n}{sp}\PY{o}{.}\PY{n}{Integral}\PY{p}{(}\PY{n}{Idiffract2}\PY{p}{,} \PY{p}{(}\PY{n}{x}\PY{p}{,}\PY{o}{\PYZhy{}}\PY{l+m+mf}{0.100}\PY{p}{,}\PY{l+m+mf}{0.100}\PY{p}{)}\PY{p}{)}\PY{o}{.}\PY{n}{evalf}\PY{p}{(}\PY{p}{)}
\end{Verbatim}

\texttt{\color{outcolor}Out[{\color{outcolor}67}]:}
    
    $$0.36$$

    

    Aha! The integral of the slit with width \(2D\) has a value that is
twice the integral of the \(D\) slit. That is what we expect since we
allow twice the amount of light through.

    \paragraph{6. You may have noticed during the experiment in the lab that
the maxima of a single-slit diffraction pattern are not equally spaced.
(a) By differentiating the light
intensity}\label{you-may-have-noticed-during-the-experiment-in-the-lab-that-the-maxima-of-a-single-slit-diffraction-pattern-are-not-equally-spaced.-a-by-differentiating-the-light-intensity}

\[ I = I_o \left[\frac{\sin\left(\beta/2\right)}{\beta/2}\right]^2\]

\paragraph{\texorpdfstring{with respect to \(\beta\) show that the
secondary maxima occur when \(\beta/2\) satisfies the relation
\(\tan\left(\beta/2\right)=\beta/2\). (b) Carefully and precisely plot
(using maple or some other plotting software) the curves \(y=\beta/2\)
and \(y=tan\left(\beta/s\right)\). From their intersections, determine
the values of \(\beta\) for the first and second secondary maxima. What
is the percent difference from \(\beta/2=(m+1/2)\pi\)
?}{with respect to \textbackslash{}beta show that the secondary maxima occur when \textbackslash{}beta/2 satisfies the relation \textbackslash{}tan\textbackslash{}left(\textbackslash{}beta/2\textbackslash{}right)=\textbackslash{}beta/2. (b) Carefully and precisely plot (using maple or some other plotting software) the curves y=\textbackslash{}beta/2 and y=tan\textbackslash{}left(\textbackslash{}beta/s\textbackslash{}right). From their intersections, determine the values of \textbackslash{}beta for the first and second secondary maxima. What is the percent difference from \textbackslash{}beta/2=(m+1/2)\textbackslash{}pi ?}}\label{with-respect-to-beta-show-that-the-secondary-maxima-occur-when-beta2-satisfies-the-relation-tanleftbeta2rightbeta2.-b-carefully-and-precisely-plot-using-maple-or-some-other-plotting-software-the-curves-ybeta2-and-ytanleftbetasright.-from-their-intersections-determine-the-values-of-beta-for-the-first-and-second-secondary-maxima.-what-is-the-percent-difference-from-beta2m12pi}

\begin{enumerate}
\def\labelenumi{(\alph{enumi})}
\tightlist
\item
  The derivative will be to differentiate the square. Then, chain rule
  the inside function of \(\beta\).
\end{enumerate}

\[\begin{equation}
\frac{dI}{d\beta} = 2I_o\left[\frac{\sin\left(\beta/2\right)}{\beta/2}\right] \cdot \left[\frac{\frac{1}{2}\cos\left(\beta/2\right)}{\beta/2}-\frac{2\sin\left(\beta/2\right)}{\left(\beta\right)^2}\right]
\\
\\
\frac{dI}{d\beta} = 2I_o\left[\frac{\sin\left(\beta/2\right)}{\beta/2}\right] \cdot \left[\frac{\cos\left(\beta/2\right)}{\beta}-\frac{2\sin\left(\beta/2\right)}{\left(\beta\right)^2}\right]
\end{equation}\]

The maxima and minima correspond to the zeros of the derivative. The
derivative can be zero when the first term is zero.
\[\left[\frac{\sin\left(\beta/2\right)}{\beta/2}\right]=0\] This will
occur when \(\beta = 2m\pi\) or \(\beta/2\) is an integer multiple of
\(\pi\). It will also be zero when \(\beta=\infty\). All of these values
are minima. See graph below.

    \begin{Verbatim}[commandchars=\\\{\}]
{\color{incolor}In [{\color{incolor}5}]:} \PY{n}{sp}\PY{o}{.}\PY{n}{plot}\PY{p}{(}\PY{n}{sp}\PY{o}{.}\PY{n}{sin}\PY{p}{(}\PY{n}{x}\PY{o}{/}\PY{l+m+mi}{2}\PY{p}{)}\PY{o}{/}\PY{p}{(}\PY{n}{x}\PY{o}{/}\PY{l+m+mi}{2}\PY{p}{)}\PY{p}{,} \PY{p}{(}\PY{n}{x}\PY{p}{,}\PY{o}{\PYZhy{}}\PY{l+m+mi}{25}\PY{p}{,}\PY{l+m+mi}{25}\PY{p}{)}\PY{p}{,} \PY{n}{xlabel}\PY{o}{=}\PY{l+s+s1}{\PYZsq{}}\PY{l+s+s1}{\PYZdl{}}\PY{l+s+se}{\PYZbs{}\PYZbs{}}\PY{l+s+s1}{beta\PYZdl{}}\PY{l+s+s1}{\PYZsq{}}\PY{p}{,} \PY{n}{ylabel}\PY{o}{=}\PY{l+s+s1}{\PYZsq{}}\PY{l+s+s1}{First Term of Derivative}\PY{l+s+s1}{\PYZsq{}}\PY{p}{)}
\end{Verbatim}


    \begin{center}
    \adjustimage{max size={0.9\linewidth}{0.9\paperheight}}{output_18_0.png}
    \end{center}
    { \hspace*{\fill} \\}
    
\begin{Verbatim}[commandchars=\\\{\}]
{\color{outcolor}Out[{\color{outcolor}5}]:} <sympy.plotting.plot.Plot at 0x7fc940488198>
\end{Verbatim}
            
    This means that the other term gives the maxima.
\[\frac{\cos\left(\beta/2\right)}{\beta}-\frac{2\sin\left(\beta/2\right)}{\left(\beta\right)^2} = 0\]
Simplifying this we get \[\begin{equation}
\frac{\cos\left(\beta/2\right)}{\beta}=\frac{2\sin\left(\beta/2\right)}{\left(\beta\right)^2}\\
\\
\frac{\beta}{2} = \frac{\sin\left(\beta/2\right)}{\cos\left(\beta/2\right)}\\
\\
\frac{\beta}{2} = \tan\left(\beta/2\right)
\end{equation}\]

    \begin{Verbatim}[commandchars=\\\{\}]
{\color{incolor}In [{\color{incolor}102}]:} \PY{n}{num6plot} \PY{o}{=} \PY{n}{sp}\PY{o}{.}\PY{n}{plot}\PY{p}{(}\PY{n}{sp}\PY{o}{.}\PY{n}{tan}\PY{p}{(}\PY{n}{x}\PY{o}{/}\PY{l+m+mi}{2}\PY{p}{)}\PY{p}{,} \PY{n}{x}\PY{o}{/}\PY{l+m+mi}{2}\PY{p}{,} \PY{p}{(}\PY{n}{x}\PY{p}{,} \PY{l+m+mi}{0}\PY{p}{,}\PY{l+m+mi}{18}\PY{p}{)}\PY{p}{,} \PY{n}{ylim}\PY{o}{=}\PY{p}{(}\PY{l+m+mi}{0}\PY{p}{,}\PY{l+m+mi}{10}\PY{p}{)}\PY{p}{,} \PY{n}{show}\PY{o}{=}\PY{k+kc}{False}\PY{p}{,} \PY{n}{legend}\PY{o}{=}\PY{k+kc}{True}\PY{p}{,} \PY{n}{ylabel}\PY{o}{=}\PY{l+s+s1}{\PYZsq{}}\PY{l+s+s1}{y}\PY{l+s+s1}{\PYZsq{}}\PY{p}{,} \PY{n}{xlabel}\PY{o}{=}\PY{l+s+s1}{\PYZsq{}}\PY{l+s+s1}{\PYZdl{}}\PY{l+s+se}{\PYZbs{}\PYZbs{}}\PY{l+s+s1}{beta\PYZdl{}}\PY{l+s+s1}{\PYZsq{}}\PY{p}{)}
          \PY{n}{num6plot}\PY{p}{[}\PY{l+m+mi}{0}\PY{p}{]}\PY{o}{.}\PY{n}{line\PYZus{}color}\PY{o}{=}\PY{l+s+s1}{\PYZsq{}}\PY{l+s+s1}{green}\PY{l+s+s1}{\PYZsq{}}
          \PY{n}{num6plot}\PY{p}{[}\PY{l+m+mi}{0}\PY{p}{]}\PY{o}{.}\PY{n}{label}\PY{o}{=}\PY{l+s+s1}{\PYZsq{}}\PY{l+s+s1}{tan(\PYZdl{}}\PY{l+s+se}{\PYZbs{}\PYZbs{}}\PY{l+s+s1}{beta\PYZdl{}/2)}\PY{l+s+s1}{\PYZsq{}}
          \PY{n}{num6plot}\PY{p}{[}\PY{l+m+mi}{1}\PY{p}{]}\PY{o}{.}\PY{n}{line\PYZus{}color}\PY{o}{=}\PY{l+s+s1}{\PYZsq{}}\PY{l+s+s1}{blue}\PY{l+s+s1}{\PYZsq{}}
          \PY{n}{num6plot}\PY{p}{[}\PY{l+m+mi}{1}\PY{p}{]}\PY{o}{.}\PY{n}{label}\PY{o}{=}\PY{l+s+s1}{\PYZsq{}}\PY{l+s+s1}{\PYZdl{}}\PY{l+s+se}{\PYZbs{}\PYZbs{}}\PY{l+s+s1}{beta\PYZdl{}/2}\PY{l+s+s1}{\PYZsq{}}
          \PY{n}{num6plot}\PY{o}{.}\PY{n}{show}\PY{p}{(}\PY{p}{)}
\end{Verbatim}


    \begin{center}
    \adjustimage{max size={0.9\linewidth}{0.9\paperheight}}{output_20_0.png}
    \end{center}
    { \hspace*{\fill} \\}
    
    There's a crossing at \(\beta=0\). We expected that since that is the
central maximum. Let's zoom in on the two crossings around \(\beta=9\)
and \(\beta =15\).

    \begin{Verbatim}[commandchars=\\\{\}]
{\color{incolor}In [{\color{incolor}11}]:} \PY{n}{num6plotz1} \PY{o}{=} \PY{n}{sp}\PY{o}{.}\PY{n}{plot}\PY{p}{(}\PY{n}{sp}\PY{o}{.}\PY{n}{tan}\PY{p}{(}\PY{n}{x}\PY{o}{/}\PY{l+m+mi}{2}\PY{p}{)}\PY{p}{,} \PY{n}{x}\PY{o}{/}\PY{l+m+mi}{2}\PY{p}{,} \PY{p}{(}\PY{n}{x}\PY{p}{,} \PY{l+m+mf}{8.98}\PY{p}{,}\PY{l+m+mf}{8.995}\PY{p}{)}\PY{p}{,} \PY{n}{ylim}\PY{o}{=}\PY{p}{(}\PY{l+m+mf}{4.4}\PY{p}{,}\PY{l+m+mf}{4.55}\PY{p}{)}\PY{p}{,} \PY{n}{show}\PY{o}{=}\PY{k+kc}{False}\PY{p}{,} \PY{n}{legend}\PY{o}{=}\PY{k+kc}{True}\PY{p}{,} \PY{n}{ylabel}\PY{o}{=}\PY{l+s+s1}{\PYZsq{}}\PY{l+s+s1}{y}\PY{l+s+s1}{\PYZsq{}}\PY{p}{,} \PY{n}{xlabel}\PY{o}{=}\PY{l+s+s1}{\PYZsq{}}\PY{l+s+s1}{\PYZdl{}}\PY{l+s+se}{\PYZbs{}\PYZbs{}}\PY{l+s+s1}{beta\PYZdl{}}\PY{l+s+s1}{\PYZsq{}}\PY{p}{)}
         \PY{n}{num6plotz1}\PY{p}{[}\PY{l+m+mi}{0}\PY{p}{]}\PY{o}{.}\PY{n}{line\PYZus{}color}\PY{o}{=}\PY{l+s+s1}{\PYZsq{}}\PY{l+s+s1}{green}\PY{l+s+s1}{\PYZsq{}}
         \PY{n}{num6plotz1}\PY{p}{[}\PY{l+m+mi}{0}\PY{p}{]}\PY{o}{.}\PY{n}{label}\PY{o}{=}\PY{l+s+s1}{\PYZsq{}}\PY{l+s+s1}{tan(\PYZdl{}}\PY{l+s+se}{\PYZbs{}\PYZbs{}}\PY{l+s+s1}{beta\PYZdl{}/2)}\PY{l+s+s1}{\PYZsq{}}
         \PY{n}{num6plotz1}\PY{p}{[}\PY{l+m+mi}{1}\PY{p}{]}\PY{o}{.}\PY{n}{line\PYZus{}color}\PY{o}{=}\PY{l+s+s1}{\PYZsq{}}\PY{l+s+s1}{blue}\PY{l+s+s1}{\PYZsq{}}
         \PY{n}{num6plotz1}\PY{p}{[}\PY{l+m+mi}{1}\PY{p}{]}\PY{o}{.}\PY{n}{label}\PY{o}{=}\PY{l+s+s1}{\PYZsq{}}\PY{l+s+s1}{\PYZdl{}}\PY{l+s+se}{\PYZbs{}\PYZbs{}}\PY{l+s+s1}{beta\PYZdl{}/2}\PY{l+s+s1}{\PYZsq{}}
         \PY{n}{num6plotz1}\PY{o}{.}\PY{n}{show}\PY{p}{(}\PY{p}{)}
\end{Verbatim}


    \begin{center}
    \adjustimage{max size={0.9\linewidth}{0.9\paperheight}}{output_22_0.png}
    \end{center}
    { \hspace*{\fill} \\}
    
    \begin{Verbatim}[commandchars=\\\{\}]
{\color{incolor}In [{\color{incolor}17}]:} \PY{n}{num6plotz2} \PY{o}{=} \PY{n}{sp}\PY{o}{.}\PY{n}{plot}\PY{p}{(}\PY{n}{sp}\PY{o}{.}\PY{n}{tan}\PY{p}{(}\PY{n}{x}\PY{o}{/}\PY{l+m+mi}{2}\PY{p}{)}\PY{p}{,} \PY{n}{x}\PY{o}{/}\PY{l+m+mi}{2}\PY{p}{,} \PY{p}{(}\PY{n}{x}\PY{p}{,} \PY{l+m+mf}{15.43}\PY{p}{,}\PY{l+m+mf}{15.47}\PY{p}{)}\PY{p}{,} \PY{n}{ylim}\PY{o}{=}\PY{p}{(}\PY{l+m+mf}{7.7}\PY{p}{,}\PY{l+m+mf}{7.75}\PY{p}{)}\PY{p}{,} \PY{n}{show}\PY{o}{=}\PY{k+kc}{False}\PY{p}{,} \PY{n}{legend}\PY{o}{=}\PY{k+kc}{True}\PY{p}{,} \PY{n}{ylabel}\PY{o}{=}\PY{l+s+s1}{\PYZsq{}}\PY{l+s+s1}{y}\PY{l+s+s1}{\PYZsq{}}\PY{p}{,} \PY{n}{xlabel}\PY{o}{=}\PY{l+s+s1}{\PYZsq{}}\PY{l+s+s1}{\PYZdl{}}\PY{l+s+se}{\PYZbs{}\PYZbs{}}\PY{l+s+s1}{beta\PYZdl{}}\PY{l+s+s1}{\PYZsq{}}\PY{p}{)}
         \PY{n}{num6plotz2}\PY{p}{[}\PY{l+m+mi}{0}\PY{p}{]}\PY{o}{.}\PY{n}{line\PYZus{}color}\PY{o}{=}\PY{l+s+s1}{\PYZsq{}}\PY{l+s+s1}{green}\PY{l+s+s1}{\PYZsq{}}
         \PY{n}{num6plotz2}\PY{p}{[}\PY{l+m+mi}{0}\PY{p}{]}\PY{o}{.}\PY{n}{label}\PY{o}{=}\PY{l+s+s1}{\PYZsq{}}\PY{l+s+s1}{tan(\PYZdl{}}\PY{l+s+se}{\PYZbs{}\PYZbs{}}\PY{l+s+s1}{beta\PYZdl{}/2)}\PY{l+s+s1}{\PYZsq{}}
         \PY{n}{num6plotz2}\PY{p}{[}\PY{l+m+mi}{1}\PY{p}{]}\PY{o}{.}\PY{n}{line\PYZus{}color}\PY{o}{=}\PY{l+s+s1}{\PYZsq{}}\PY{l+s+s1}{blue}\PY{l+s+s1}{\PYZsq{}}
         \PY{n}{num6plotz2}\PY{p}{[}\PY{l+m+mi}{1}\PY{p}{]}\PY{o}{.}\PY{n}{label}\PY{o}{=}\PY{l+s+s1}{\PYZsq{}}\PY{l+s+s1}{\PYZdl{}}\PY{l+s+se}{\PYZbs{}\PYZbs{}}\PY{l+s+s1}{beta\PYZdl{}/2}\PY{l+s+s1}{\PYZsq{}}
         \PY{n}{num6plotz2}\PY{o}{.}\PY{n}{show}\PY{p}{(}\PY{p}{)}
\end{Verbatim}


    \begin{center}
    \adjustimage{max size={0.9\linewidth}{0.9\paperheight}}{output_23_0.png}
    \end{center}
    { \hspace*{\fill} \\}
    
    It appears the first maximum is when \(\beta=8.987\) radians, and the
second maximum is when \(\beta=15.451\) radians. We expect
\(\frac{\beta}{2}=\left(m+\frac{1}{2}\right)\pi\). We should get the
first three maxima at \(m = 1\) and \(m = 2\).

\[\beta = \left(2m+1\right)\pi\]

\[\beta = 3\pi = 9.425~\rm{rad}\]

\[\beta = 5\pi = 15.71~\rm{rad}\]

The percent errors from being evenly spaced maxima are

\[\frac{\Delta\beta}{\beta_1} = \frac{9.425-8.987}{9.425} = 0.046 = 4.6\%\]

\[\frac{\Delta\beta}{\beta_2} = \frac{15.71-15.451}{15.71} = 0.016 = 1.6\%\]

    \paragraph{\texorpdfstring{7. A diffraction grating is an optical
component with a periodic structure, which splits and diffracts light
into several beams travelling in different directions. The CD grooves
act as a diffraction grating and break down the light into component
wavelengths. Imagine a CD that is designed to have the second-order
maxima at \(9.5^{\circ}\) from the central maximum for the red end
(\(\lambda=675\) nm) of the visible spectrum. How many lines per
centimeter does the grating
have?}{7. A diffraction grating is an optical component with a periodic structure, which splits and diffracts light into several beams travelling in different directions. The CD grooves act as a diffraction grating and break down the light into component wavelengths. Imagine a CD that is designed to have the second-order maxima at 9.5\^{}\{\textbackslash{}circ\} from the central maximum for the red end (\textbackslash{}lambda=675 nm) of the visible spectrum. How many lines per centimeter does the grating have?}}\label{a-diffraction-grating-is-an-optical-component-with-a-periodic-structure-which-splits-and-diffracts-light-into-several-beams-travelling-in-different-directions.-the-cd-grooves-act-as-a-diffraction-grating-and-break-down-the-light-into-component-wavelengths.-imagine-a-cd-that-is-designed-to-have-the-second-order-maxima-at-9.5circ-from-the-central-maximum-for-the-red-end-lambda675-nm-of-the-visible-spectrum.-how-many-lines-per-centimeter-does-the-grating-have}

\[\begin{equation}
d \sin\theta = m \lambda \\
\\
d = \frac{2\left(675\times{-9}\right)}{\sin9.5^{\circ}} \\
\\
d = 8.18\times10^{-6} \rm{m}\\
\end{equation}
\]

    \paragraph{\texorpdfstring{8. When multiple wavelengths are reflected by
a diffraction gratings, the different diffraction order can start
overlapping. It is found that the violet (\(\lambda=4.00\times 10^2\)
nm) portion of the third-order maximum overlaps the yellow-orange
(\(\lambda=6.00\times10^2\) nm) portion of the second-order maximum when
the gratings spacing is \(1.2\times10^5\) lines per cm. Show that this
would be the case for a diffraction grating, regardless of the grating's
spacing.}{8. When multiple wavelengths are reflected by a diffraction gratings, the different diffraction order can start overlapping. It is found that the violet (\textbackslash{}lambda=4.00\textbackslash{}times 10\^{}2 nm) portion of the third-order maximum overlaps the yellow-orange (\textbackslash{}lambda=6.00\textbackslash{}times10\^{}2 nm) portion of the second-order maximum when the gratings spacing is 1.2\textbackslash{}times10\^{}5 lines per cm. Show that this would be the case for a diffraction grating, regardless of the grating's spacing.}}\label{when-multiple-wavelengths-are-reflected-by-a-diffraction-gratings-the-different-diffraction-order-can-start-overlapping.-it-is-found-that-the-violet-lambda4.00times-102-nm-portion-of-the-third-order-maximum-overlaps-the-yellow-orange-lambda6.00times102-nm-portion-of-the-second-order-maximum-when-the-gratings-spacing-is-1.2times105-lines-per-cm.-show-that-this-would-be-the-case-for-a-diffraction-grating-regardless-of-the-gratings-spacing.}

\begin{figure}[htbp]
\centering
\includegraphics{attachment:image.png}
\caption{image.png}
\end{figure}

Using the diffraction equation \[d \sin\theta = m\lambda\] we can plug
in \(m = 2\) for \(\lambda=6.00\times10^{-9}\) m to get
\[m\lambda = 1.2\times10^{-8} \rm{m}\] Similarly, plugging in \(m = 3\)
for \(\lambda=4.00\times10^{-9}\) m to get
\[m\lambda = 1.2\times10^{-8} \rm{m}\] This means \(d\sin\theta\) has
the same constructive condition no matter what \(d\) is used (as long as
it is the same for both wavelengths).

    \paragraph{\texorpdfstring{9. Potassium chloride (KCl) has a set of
crystal planes separated by a distance \(d=0.31\) nm (1/2 of what is
shown as ``\(a\)'' in the figure below). At what angle \(\theta\) to
these planes would the first-order Bragg maximum occur? Assume the
x-rays are from the \(K\alpha\) line of copper with a wavelength of
\(\lambda=1.5406\)
Angstrom.}{9. Potassium chloride (KCl) has a set of crystal planes separated by a distance d=0.31 nm (1/2 of what is shown as ``a'' in the figure below). At what angle \textbackslash{}theta to these planes would the first-order Bragg maximum occur? Assume the x-rays are from the K\textbackslash{}alpha line of copper with a wavelength of \textbackslash{}lambda=1.5406 Angstrom.}}\label{potassium-chloride-kcl-has-a-set-of-crystal-planes-separated-by-a-distance-d0.31-nm-12-of-what-is-shown-as-a-in-the-figure-below.-at-what-angle-theta-to-these-planes-would-the-first-order-bragg-maximum-occur-assume-the-x-rays-are-from-the-kalpha-line-of-copper-with-a-wavelength-of-lambda1.5406-angstrom.}

\begin{figure}[htbp]
\centering
\includegraphics{attachment:image.png}
\caption{image.png}
\end{figure}

This is an interference problem dealing with reflection from parallel
planes of atoms. Therefore, we use Bragg's Law.

\[\begin{equation}
2 d\sin\theta = m\lambda\\
\\
2\left(0.31\times10^{-9}\right)\sin\theta=1\left(1.5406\times10^{-10}\right)\\
\\
\sin\theta = 0.24848\\
\\
\theta = \sin^{-1}\left(0.24848\right)\\
\\
\theta = 14.39^{\circ}
\end{equation}\]

I have some x-ray diffraction data taken with our spectrometer. Let's
take a look.

    \begin{Verbatim}[commandchars=\\\{\}]
{\color{incolor}In [{\color{incolor}41}]:} \PY{k+kn}{import} \PY{n+nn}{pandas} \PY{k}{as} \PY{n+nn}{pd}
         \PY{k+kn}{import} \PY{n+nn}{matplotlib}\PY{n+nn}{.}\PY{n+nn}{pyplot} \PY{k}{as} \PY{n+nn}{plt}
         \PY{n}{df} \PY{o}{=} \PY{n}{pd}\PY{o}{.}\PY{n}{read\PYZus{}excel}\PY{p}{(}\PY{n+nb}{open}\PY{p}{(}\PY{l+s+s1}{\PYZsq{}}\PY{l+s+s1}{/home/nbuser/library/KCl\PYZhy{}XRD.xlsx}\PY{l+s+s1}{\PYZsq{}}\PY{p}{,}\PY{l+s+s1}{\PYZsq{}}\PY{l+s+s1}{rb}\PY{l+s+s1}{\PYZsq{}}\PY{p}{)}\PY{p}{,} \PY{n}{sheetname}\PY{o}{=}\PY{l+m+mi}{0}\PY{p}{,} \PY{n}{parse\PYZus{}cols} \PY{o}{=} \PY{l+s+s2}{\PYZdq{}}\PY{l+s+s2}{B:C}\PY{l+s+s2}{\PYZdq{}}\PY{p}{)}
         \PY{c+c1}{\PYZsh{}sp.plot(df[\PYZsq{}Intensity\PYZsq{}],df[\PYZsq{}theta\PYZsq{}])}
         \PY{n}{angle}\PY{o}{=}\PY{n}{df}\PY{p}{[}\PY{l+s+s1}{\PYZsq{}}\PY{l+s+s1}{theta}\PY{l+s+s1}{\PYZsq{}}\PY{p}{]}\PY{o}{.}\PY{n}{tolist}\PY{p}{(}\PY{p}{)}
         \PY{n}{Intense}\PY{o}{=}\PY{n}{df}\PY{p}{[}\PY{l+s+s1}{\PYZsq{}}\PY{l+s+s1}{Intensity}\PY{l+s+s1}{\PYZsq{}}\PY{p}{]}\PY{o}{.}\PY{n}{tolist}\PY{p}{(}\PY{p}{)}
         \PY{n}{plt}\PY{o}{.}\PY{n}{plot}\PY{p}{(}\PY{n}{angle}\PY{p}{,} \PY{n}{Intense}\PY{p}{)}
         \PY{n}{plt}\PY{o}{.}\PY{n}{ylabel}\PY{p}{(}\PY{l+s+s1}{\PYZsq{}}\PY{l+s+s1}{Intensity}\PY{l+s+s1}{\PYZsq{}}\PY{p}{)}
         \PY{n}{plt}\PY{o}{.}\PY{n}{xlabel}\PY{p}{(}\PY{l+s+s1}{\PYZsq{}}\PY{l+s+s1}{\PYZdl{}}\PY{l+s+se}{\PYZbs{}\PYZbs{}}\PY{l+s+s1}{theta\PYZdl{}}\PY{l+s+s1}{\PYZsq{}}\PY{p}{)}
\end{Verbatim}


\begin{Verbatim}[commandchars=\\\{\}]
{\color{outcolor}Out[{\color{outcolor}41}]:} Text(0.5,0,'\$\textbackslash{}\textbackslash{}theta\$')
\end{Verbatim}
            
    \begin{center}
    \adjustimage{max size={0.9\linewidth}{0.9\paperheight}}{output_28_1.png}
    \end{center}
    { \hspace*{\fill} \\}
    
    \paragraph{Bonus 1: Two narrow slits are separated by 0.48 mm. The slits
are illuminated with a monochromatic light. The pattern of interference
on the screen ( 5.0 m away from the slit) is shown in the figure below.
It shows the intensity of the light on the y-axis and the distance from
the central maximum on the screen on the x-axis. (a) What would be the
maximum intensity of the light falling on the screen if only one slit
were open? (b) Find the wavelength of the light based on the diffraction
pattern
below.}\label{bonus-1-two-narrow-slits-are-separated-by-0.48-mm.-the-slits-are-illuminated-with-a-monochromatic-light.-the-pattern-of-interference-on-the-screen-5.0-m-away-from-the-slit-is-shown-in-the-figure-below.-it-shows-the-intensity-of-the-light-on-the-y-axis-and-the-distance-from-the-central-maximum-on-the-screen-on-the-x-axis.-a-what-would-be-the-maximum-intensity-of-the-light-falling-on-the-screen-if-only-one-slit-were-open-b-find-the-wavelength-of-the-light-based-on-the-diffraction-pattern-below.}

\begin{figure}[htbp]
\centering
\includegraphics{attachment:image.png}
\caption{image.png}
\end{figure}

    \paragraph{\texorpdfstring{Bonus 2: Let's recreate Young's double slit
experiment with a monochromatic light of wavelength \(\lambda\) of 500
nm. The two thin slits are 0.1 mm apart. You would like to study the
effect of a filter on one of the slit. The filter dims the light
intensity through the slit by a factor of two compared to the other
slit. Determine the intensity \(I\) as a function of angular position
\(\theta\) on the
screen.}{Bonus 2: Let's recreate Young's double slit experiment with a monochromatic light of wavelength \textbackslash{}lambda of 500 nm. The two thin slits are 0.1 mm apart. You would like to study the effect of a filter on one of the slit. The filter dims the light intensity through the slit by a factor of two compared to the other slit. Determine the intensity I as a function of angular position \textbackslash{}theta on the screen.}}\label{bonus-2-lets-recreate-youngs-double-slit-experiment-with-a-monochromatic-light-of-wavelength-lambda-of-500-nm.-the-two-thin-slits-are-0.1-mm-apart.-you-would-like-to-study-the-effect-of-a-filter-on-one-of-the-slit.-the-filter-dims-the-light-intensity-through-the-slit-by-a-factor-of-two-compared-to-the-other-slit.-determine-the-intensity-i-as-a-function-of-angular-position-theta-on-the-screen.}


    % Add a bibliography block to the postdoc
    
    
    
    \end{document}
